% =====================================================
% CHƯƠNG 4: KẾT QUẢ VÀ THẢO LUẬN
% Sinh viên phụ trách: [Tên sinh viên]
% =====================================================

\chapter{Kết quả và thảo luận}

\section{Kết quả thực nghiệm}

\subsection{Thực nghiệm 1}
Mô tả và kết quả của thực nghiệm thứ nhất... 

\begin{table}[H]
\centering
\caption{Kết quả thực nghiệm 1}
\label{tab:exp1}
\renewcommand{\arraystretch}{1.5}
\begin{tabular}{|c|c|c|c|}
\hline
\textbf{STT} & \textbf{Tham số} & \textbf{Giá trị} & \textbf{Đơn vị} \\ \hline
1 & Độ chính xác & 95.5 & \% \\ \hline
2 & Thời gian xử lý & 1.2 & giây \\ \hline
3 & Bộ nhớ sử dụng & 512 & MB \\ \hline
\end{tabular}
\end{table}

Như thể hiện trong Bảng \ref{tab:exp1}, kết quả cho thấy... 

\subsection{Thực nghiệm 2}
Mô tả và kết quả của thực nghiệm thứ hai... 

\begin{figure}[H]
    \centering
    \includegraphics[width=0.8\textwidth]{example-image}
    \caption{Biểu đồ so sánh kết quả}
    \label{fig:comparison}
\end{figure}

\section{So sánh với các phương pháp khác}

\begin{table}[H]
\centering
\caption{So sánh hiệu suất các phương pháp}
\label{tab:comparison}
\renewcommand{\arraystretch}{1.5}
\begin{tabular}{|l|c|c|c|}
\hline
\textbf{Phương pháp} & \textbf{Độ chính xác (\%)} & \textbf{Thời gian (s)} & \textbf{Bộ nhớ (MB)} \\ \hline
Phương pháp đề xuất & \textbf{95.5} & \textbf{1.2} & 512 \\ \hline
Phương pháp A & 92.3 & 1.5 & 480 \\ \hline
Phương pháp B & 93.8 & 1.8 & 520 \\ \hline
Phương pháp C & 91.5 & 1.0 & \textbf{450} \\ \hline
\end{tabular}
\end{table}

\section{Thảo luận}

\subsection{Phân tích kết quả}
Từ các kết quả thu được, ta có thể thấy rằng...

\subsection{Ưu điểm}
Phương pháp đề xuất có các ưu điểm sau:
\begin{enumerate}
    \item Độ chính xác cao hơn các phương pháp khác
    \item Thời gian xử lý nhanh
    \item Dễ dàng triển khai
\end{enumerate}

\subsection{Nhược điểm}
Tuy nhiên, phương pháp vẫn còn một số hạn chế:
\begin{enumerate}
    \item Sử dụng nhiều bộ nhớ hơn
    \item Chưa tối ưu cho dữ liệu lớn
    \item Cần cải thiện giao diện người dùng
\end{enumerate}

\section{Một số hình ảnh minh họa}

\begin{figure}[H]
    \centering
    \begin{subfigure}[b]{0.45\textwidth}
        \includegraphics[width=\textwidth]{example-image}
        \caption{Kết quả 1}
        \label{fig:result1}
    \end{subfigure}
    \hfill
    \begin{subfigure}[b]{0.45\textwidth}
        \includegraphics[width=\textwidth]{example-image}
        \caption{Kết quả 2}
        \label{fig:result2}
    \end{subfigure}
    \caption{So sánh kết quả trực quan}
    \label{fig:results}
\end{figure}