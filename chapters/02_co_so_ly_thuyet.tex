% =====================================================
% CHƯƠNG 2: CƠ SỞ LÝ THUYẾT
% Sinh viên phụ trách: [Tên sinh viên]
% =====================================================

\chapter{Cơ sở lý thuyết}

\section{Tổng quan}
Chương này trình bày các khái niệm cơ bản và lý thuyết nền tảng... 

\section{Khái niệm cơ bản}

\subsection{Khái niệm 1}
Định nghĩa và giải thích khái niệm 1...

\subsection{Khái niệm 2}
Định nghĩa và giải thích khái niệm 2...

\section{Các công thức toán học}

Ví dụ về công thức trong dòng: $E = mc^2$

Công thức riêng biệt:
\begin{equation}
    \int_{a}^{b} f(x)dx = F(b) - F(a)
    \label{eq:integral}
\end{equation}

Hệ phương trình:
\begin{align}
    x + y &= 5 \label{eq:eq1}\\
    2x - y &= 1 \label{eq:eq2}
\end{align}

Từ phương trình \eqref{eq:eq1} và \eqref{eq:eq2}, ta có thể giải được... 

\section{Mô hình lý thuyết}

\begin{figure}[H]
    \centering
    \includegraphics[width=0.7\textwidth]{example-image}
    \caption{Mô hình lý thuyết}
    \label{fig:model}
\end{figure}

Như thể hiện trong Hình \ref{fig:model}, mô hình bao gồm... 

\section{Các nghiên cứu liên quan}
Các nghiên cứu liên quan bao gồm \cite{kien_truc_pm_2020,deep_learning_2015}.
