% ===== LỀ TRANG =====
\geometry{
  a4paper,
  top=2.5cm,
  bottom=2.5cm,
  left=2.5cm, % Sử dụng 3cm cho bản in hai mặt
  right=2cm
}

% ===== MỤC LỤC =====
\setlength{\cftchapnumwidth}{3em}

% ===== CÀI ĐẶT CHO CODE =====
\lstdefinestyle{mystyle}{
    backgroundcolor=\color{gray!10},
    commentstyle=\color{green!50!black},
    keywordstyle=\color{blue},
    numberstyle=\tiny\color{gray},
    stringstyle=\color{orange},
    basicstyle=\ttfamily\small,
    breakatwhitespace=false,
    breaklines=true,
    captionpos=b,
    keepspaces=true,
    numbers=none,
    showspaces=false,
    showstringspaces=false,
    showtabs=false,
    tabsize=2,
    frame=single,
}
\lstset{style=mystyle}

% ===== KHOẢNG CÁCH DÒNG =====
\onehalfspacing

% ===== ĐỊNH DẠNG TIÊU ĐỀ =====
% Chuyển tiêu đề chương sang số la mã
\renewcommand{\thechapter}{\Roman{chapter}}
% Chuyển tiêu đề chương thành 1 dòng
\titleformat{\chapter}
  {\normalfont\huge\bfseries}
  {Chương \thechapter}
  {0.75em}
  {}
% Loại bỏ chương khi đánh số phần
% Ví dụ: I.1.1 -> 1.1
\renewcommand{\thesection}{\arabic{section}}
\renewcommand{\thesubsection}{\arabic{section}.\arabic{subsection}}
\renewcommand{\thesubsubsection}{\arabic{section}.\arabic{subsection}.\arabic{subsubsection}}

% ===== HEADER =====
\pagestyle{fancy}
\fancyhead{}
\fancyhead[L]{\nouppercase{\leftmark}}
% Sử dụng dòng dưới cho bản in hai mặt
% \fancyhead[RE,LO]{\nouppercase{\leftmark}}
\renewcommand{\headrulewidth}{0.2pt}
\setlength{\headheight}{14.3pt}